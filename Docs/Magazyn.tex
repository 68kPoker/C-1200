\documentclass[12pt]{article}
\author{Robert Szacki}
\title{Game for Amiga OS - Sokoban clone with advanced GUI}
\begin{document}
\maketitle
\section{OS Init}

Some initiation must be performed at first place. These are shared Amiga libraries and devices.

\section{Data}

All data are loaded from Interchange File Format files. This includes text, graphics, sounds and music. Game data is stored in own chunks.

The graphics is used to decide about screen palette.

\section{Screen}

\subsection{Single or double buffering?}

We will use single buffering as it's easier to maintain, especially with intuition gadgets. But we will use
additional buffer to draw intermediate objects and then draw all differences on the screen.

\subsection{Animation frames}

Animation frames will be handled with VBlank or Copper interrupt. The main loop will send signals to windows
about the new animation frame, so they can update their graphics.

\section{Windows}

\subsection{Dragging}

For windows dragging we will use custom handler executed upon every vertical blank or Copper interrupt. This handler
draws the drag-border on window's MouseX and MouseY position.

\begin{enumerate}
\item If the drag-bar is hit with LMB, save the window border graphics and mouse position then draw drag-border.
\item Every VBlank/Copper interrupt if mouse position has changed, restore graphics, save it and draw in new
 position.
\item Once released, the window border is restored and the window is moved to new position.
\end{enumerate}

Data will be saved in Screen's UserData.

The prefered method for saving graphics will be ReadPixel() for vertical and ReadPixelLine8() for horizontal 
border and store it in pixel array. Then optional mapping and WritePixelLine8().

\subsection{Moving}

For windows movement using custom rendering we will use opening of the window in new position, closing the window 
in previous position and restoring graphics directly using layers.

\section{Gadgets}

We will use standard intuition and BOOPSI gadgets with custom imagery. Some of the gadget types need to be designed.

\subsection{Buttons}

Buttons will be implemented with intuition boolean gadgets with custom imagery. The imagery will be drawn manually or by BOOPSI image class, so we can provide support for interleaved bitmaps.

\subsection{List view gadget}

The ,,list view'' gadget will be simply a gadget containing a list of sub-gadgets of any kind in a scrollable area.

We need to store Top, Visible and Total values (we can use proportional gadget). Then we can detect the selected gadget based on Top, and send all signals to it.

\section{Graphics}

The graphics will be drawn preferably with Blitter directly or custom chunky-to-planar routine.

Aligned blits will be accelerated.

\subsection{Tiles}

Tiles will be used preferably with aligned blits. Additionally there are single and dual-layer tiles.

\subsection{Blitter Objects}

Bobs will be drawn with cookie-cut operation. The background graphics will be stored in special buffer.

The draw operation on active Bobs is executed in following order:
\begin{enumerate}
\item \label{restore_Back} The background under the Bobs is restored.
\item \label{draw_Back} The new background is drawn.
\item The background is saved.
\item \label{draw_Bobs} The Bobs are drawn.
\end{enumerate}
To remove a Bob, after phase \ref{restore_Back}. the rest is skipped. To put Bob in a drawing area, once drawn - the Bob is deactivated.

The phase \ref{draw_Back}. can be drawn throguh Stencil if we wish to preserve some foreground objects that are not actively moved in the display.

\end{document}
